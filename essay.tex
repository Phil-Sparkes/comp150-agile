% Please do not change the document class
\documentclass{scrartcl}

% Please do not change these packages
\usepackage[hidelinks]{hyperref}
\usepackage[none]{hyphenat}
\usepackage{setspace}
\doublespace

% You may add additional packages here
\usepackage{amsmath}

% Please include a clear, concise, and descriptive title
\title{How can you overcome the absence of critical team members in an agile way?}

% Please do not change the subtitle
\subtitle{COMP150 - Agile Development Practice}

% Please put your student number in the author field
\author{1608305}

\begin{document}

\maketitle

\abstract{Please include an abstract of at most 100 words (these do not count towards your word count).}

\section{Introduction}

In this report I will cover the topic of absence in an agile environment and how it affects a team differently from conventional waterfall methods. The way I shall go about researching this is by looking at the problems with the agile philosophy and problems with absence in general. I found this topic to be of interest because of the adaptability of agile and whether this adaptability translates to such things as absence and staff shifting. How teams can overcome the problem of absence with relation to the agile way.  I also intend to look at daily stand-up meetings and conduct research on problems with these meetings. A couple of issues here include lateness, different locations of developers, developers working from home or the meetings not even taking place at all.

\section{Agile}

The agile philosophy is all about building projects around motivated individuals [8]. The developers have more of a voice when it comes to the project and are expected to self-organise. The developers can often pick the tasks they want to work on from a backlog and see what other developers are working on with the help of sprint task boards. If someone is called away or absent, the remaining work can be spread out between the remaining developers. 
The agile method relies a lot on trust on the individuals in the team that they will get the work done. This trust is built overtime as the team gets to know each other. When staff rotations occur it will take a while for new members to integrate into the current system.  A study “highlights the role of staff rotation as when specifications are neither stable nor perfect, the tacit knowledge and understanding of personnel becomes the key.” [5] Knowing the individual skills on the team will provide a better idea of the tasks that can be completed and the time required to complete them. 
Agile teams are usually quite small so absence of team members could have a big effect on the team size, with smaller teams comes smaller productivity, finishing a sprint on time could become a problem also they might encounter skill constraints.
Organizations have encountered a lot of challenges when trying to adopt agile, the switch from traditional methods to agile methods can take a while and present a lot of problems.  The largest impediment to the adoption is people [6]. To get agile functioning correctly the people have to be acceptable to change and ready to try a different approach. Understanding of the philosophy will also help teams see why and how it works. 

\section{Stand-up meetings}
Daily scrum meetings are one of the key principles of agile scrum. These meetings are meant to be short 15 minute stand up meetings. If people cannot be there in person then often electronic communications are set up. Some teams will not be based in the same place so electronic communications are a must. Scrums meetings can be split up into different meetings for example one paper talked about a team that had members in both India and Denmark, on location both teams conduct their own scrum meetings which is later followed by a scrum of scrums in which the scrum masters would communicate conferring status of their teams. [3]
Agile scrum daily stand up meetings expose problems that people are currently having and as a team try to find a solution to these problems, absence of a team member could be a problem that would be brought up in one of these meetings. Having these meetings daily could mean that the team will be able to acknowledge this problem and adapt faster than using different methods.  On the other hand not having that person in the stand-up meeting would mean that their progress and blockers will not be reported. The team could ignore the fact that a team member is not present and thus not make any plans to adapt.
In some cases the member that is absent is required to e-mail the scrum master to give status which the scrum master would convey to the rest of the team during the stand-up meetings. But other teams have no contingency plans when this takes place. [2]
Some teams do not know why certain processes are important relating to agile. They do these processes just because the documentation tells them to. [2] Stand-up meetings are one of the processes that many people do not have the correct understanding of so they are often conducted incorrectly.
Not just absence is a problem but lateness to the stand-up meetings; some ways of addressing this have been discussed in other papers like scheduling the meeting at an off time “if the meeting starts at 9:07 a.m. emphasizes the importance of being on time.” [2]
“The best meeting is a group of three with one person sick and another out of town.” Kayser [7]
A paper on meetings found that not having the “right” people is one of the leading causes of unproductive meetings. [4] With agile meetings the whole team is usually required to attend the stand-up meetings. In one study “The team members perceived the information given in the daily meetings often as irrelevant. We believe that this was partly due to too many people attending the meetings.”[2] Some members consider the meetings as a hindrance while others find them a productive use of their time.

\section{Conclusion}

To solve the problem you must be aware of the problem. With absence a team using the agile scrum methodology should be able to pick up on others absence during the stand-up meetings. The scrum master would be able to bring up the topic and together the team would be able to discuss a plan to adapt. Also getting the absent member to email in disclosing information on their status means that the team has a better idea of the current situation and what needs to be done as a result.

\bibliographystyle{ieeetran}
\bibliography{references}

\end{document}
